\documentclass[hyperref={pdfpagelabels=false}]{beamer}

\usepackage[english]{babel}
\usepackage{xcolor}
\usepackage{lmodern}
\usepackage{amssymb}
\usepackage[makeroom]{cancel} %for crossing symbols
%\usepackage{calligra}
%\DeclareMathAlphabet{\mathcalligra}{T1}{calligra}{m}{n} %For small \mathcal letters
\makeatletter
\DeclareFontEncoding{LS1}{}{}
\DeclareFontSubstitution{LS1}{stix}{m}{n}
\DeclareMathAlphabet{\mathKel}{LS1}{stixscr}{m}{n}
\DeclareMathAlphabet{\mathcal}{LS1}{stixscr}{m}{n}
\usepackage{amsthm}
\usepackage{amsmath}
%\usepackage{mathabx}
\usepackage{stmaryrd}
\usepackage{amsbsy}
\usepackage{dsfont}
\usepackage{mathtools} %für mathclap und coloneqq
%\usepackage{amsbsy}
\usepackage{mleftright} %Distanz zu \left \right weg
\usepackage{tikz-cd}

\usepackage{tabularx} %Automatic line break of tables using X instead c l r
%\usepackage{longtable} %table auf mehreren Seiten
%\usepackage{ltxtable} %Combination of both above
\usepackage{xcolor, colortbl}

%Für die ganzen Diagramme
\usepackage{pgfplots}
\usepackage{graphicx} %Für raisebox, vertical displacement of figures
\usetikzlibrary{decorations.markings}

\definecolor{Gray}{gray}{0.85}
%\usepackage[style=authortitle-icomp]{biblatex}
%\usepackage[babel,german=guillemets]{csquotes}

%\setcounter{tocdepth}{4}
%\setcounter{tocdepth}{5}
%\setcounter{secnumdepth}{4}
%\setcounter{secnumdepth}{5}
\usepackage[backend=biber, style=numeric]{biblatex}
\addbibresource{Literatur.bib}
\newcommand{\footlineextra}[1]{
    \begin{tikzpicture}[remember picture,overlay]
        \node[yshift=2ex,anchor=south west] at (current page.south west) {\usebeamerfont{author in head/foot}\hspace{2ex}#1};
    \end{tikzpicture}
}
%\DeclareMathOperator{\sAut}{\mathKel{A\mkern-5.5mu u\mkern-4mu t\mkern-1.5mu}}
%\DeclareMathOperator{\saut}{\mathKel{a\mkern-4.5mu u\mkern-4mu t\mkern-1.5mu}}
%\DeclareMathOperator{\sEnd}{\mathKel{E\mkern-4mu n\mkern-4.5mu d\mkern-1mu}}
%\setbeamertemplate{bibliography entry title}{}
%\setbeamertemplate{bibliography entry location}{}
%\setbeamertemplate{bibliography entry note}{}
%\setbeamertemplate{bibliography item}{\insertbiblabel}

%\pagestyle{headings}
%
%\makeatletter
%\patchcmd{\beamer@calculateheadfoot}{\advance\footheight by 5pt}{\advance\footheight by 20pt}{}{}
%\makeatother

%\makeatletter
%\setbeamertemplate{footline}
%{
  %\leavevmode%
  %\hbox{%
  %\begin{beamercolorbox}[wd=.175\paperwidth,ht=2.25ex,dp=1ex,center]{author in head/foot}%
    %\usebeamerfont{author in head/foot}\insertshortauthor\expandafter\beamer@ifempty\expandafter{\beamer@shortinstitute}{}{~~(\insertshortinstitute)}
  %\end{beamercolorbox}%
  %\begin{beamercolorbox}[wd=.65\paperwidth,ht=2.25ex,dp=1ex,center]{title in head/foot}%
    %\usebeamerfont{title in head/foot}\insertshorttitle
  %\end{beamercolorbox}%
  %\begin{beamercolorbox}[wd=.175\paperwidth,ht=2.25ex,dp=1ex,right]{date in head/foot}%
    %\usebeamerfont{date in head/foot}\insertshortdate{}\hspace*{2em}
    %%\insertframenumber{} 
		%%/ \inserttotalframenumber\hspace*{2ex} 
  %\end{beamercolorbox}}%
  %\vskip0pt%
%}
%\makeatother 

%\setbeamercolor{footlinecolor}{fg=white,bg=Navyblue}
\newcommand\insertreferences{}
\setbeamertemplate{footline}{%
  \leavevmode%
  \hbox{%
  \begin{beamercolorbox}[wd=.09\paperwidth, ht=5ex,dp=1ex,center, sep=1.4ex]{author in head/foot}%
    \usebeamerfont{author in head/foot}
		%\vfill
		%Sources
		%\vfill
		Sources
  \end{beamercolorbox}%
  \begin{beamercolorbox}[wd=.91\paperwidth,ht=5ex,dp=1ex,center]{title in head/foot}%
    \usebeamerfont{title in head/foot}
    \insertreferences

  \end{beamercolorbox}
}
}

%\setbeamertemplate{footline}
%{
  %\leavevmode%
  %\hbox{%
  %\begin{beamercolorbox}[wd=.09\paperwidth,ht=4.35ex,dp=20ex,center]{author in head/foot}%
    %\usebeamerfont{author in head/foot}
		%Source
  %\end{beamercolorbox}%
  %\begin{beamercolorbox}[wd=.91\paperwidth,ht=4.35ex,dp=20ex,center]{title in head/foot}%
    %\usebeamerfont{title in head/foot}
		%Camilo Arias Abad, Marius Crainic. Representations up to homotopy of Lie algebroids. \textit{Journal für die reine und angewandte Mathematik (Crelles Journal)}, 2012(663):91–126, 2012.
  %\end{beamercolorbox}%
  %%\begin{beamercolorbox}[wd=.175\paperwidth,ht=2.25ex,dp=1ex,right]{date in head/foot}%
    %%\usebeamerfont{date in head/foot}\insertshortdate{}\hspace*{2em}
    %%%\insertframenumber{} 
		%%%/ \inserttotalframenumber\hspace*{2ex} 
  %%\end{beamercolorbox}
	%}%
  %\vskip0pt%
%}

\title{Geometry of curved Yang-Mills-Higgs gauge theories}   
%\title{Géométrie des théories de jauge courbes de Yang-Mills-Higgs}   
\author{Simon-Raphael Fischer} 
\date{23 April 2022} 
%\date{Le lundi 31 mai 2021} 

% zusaetzlich ist das usepackage{beamerthemeshadow} eingebunden 
%\usepackage{beamerthemeIlmenau}
%\usepackage{beamerthemeshadow}
\usepackage{beamerthemeDarmstadt}

%  \beamersetuncovermixins{\opaqueness<1>{25}}{\opaqueness<2->{15}}
%  sorgt dafuer das die Elemente die erst noch (zukuenftig) kommen 
%  nur schwach angedeutet erscheinen 
\beamersetuncovermixins{\opaqueness<1>{25}}{\opaqueness<2->{15}}
% klappt auch bei Tabellen, wenn teTeX verwendent wird\ldots

\beamertemplatenavigationsymbolsempty %Damit sind die kleinen Navigationssymbole unten weg

%\usesectionheadtemplate{}{}
%\usesubsectionheadtemplate{}{}

\def\be{\begin{equation}}
\def\ee{\end{equation}}
\def\bs{\begin{subequations}}
\def\es{\end{subequations}}
\def\ba#1\ea{\begin{align}#1\end{align}}
\def\bes{\begin{equation*}}
\def\ees{\end{equation*}}
\def\bas#1\eas{\begin{align*}#1\end{align*}}


\renewcommand{\qedsymbol}{}
\theoremstyle{plain}
\newtheorem{conjecture}[theorem]{Conjecture}
\newtheorem{proposition}[theorem]{Proposition}
%\newtheorem{definition}[theorem]{Definition}
\theoremstyle{remark}
\newtheorem*{remark}{Remarks}
\newtheorem*{idea}{Idea}
\newtheorem*{motivation}{Motivation}
\newtheorem*{summary}{Summary}
\newtheorem*{situation}{Situation}
\newtheorem*{lab}{Situation: Lie algebra bundles}
\newtheorem*{question}{Question}
\newtheorem*{fieldredefinition}{Field Redefinition}
\newtheorem*{construction}{Construction}
\newtheorem*{aim}{Aim}

%\theoremstyle{definition}
%\newtheorem{definition}[theorem]{Definition}
%\newtheorem*{SecondIn}{Second Inequality}

%mathrm mit mathup ersetzen, damit die font passt
\renewcommand\familydefault{\sfdefault} %comment to see the difference
\DeclareMathAlphabet      {\mathup}{OT1}{\familydefault}{m}{n}


\begin{document}


\begin{frame}
\thispagestyle{empty}
\titlepage
\end{frame} 


{
\setbeamertemplate{footline}{}
\begin{frame}
\frametitle{Table of contents}
\tableofcontents
\end{frame} 
}
\section{Curved Yang-Mills-Higgs gauge theory}
\subsection{Motivation}
{
\setbeamertemplate{footline}{}
\begin{frame}{\textbf{Infinitesimal} gauge theory}
\centering
%\begin{tikzpicture}[>=latex]
%\coordinate (a) at (0,0);
%\coordinate (b) at (4,0);
%\coordinate (c) at (0,-3);
%\draw (-2,-3) node[left] {Lie algebra $\mathfrak{g}$}--(2,-3);
%%\draw[dashed,->,postaction={
    %%decorate,
    %%decoration={
        %%markings,
        %%mark=at position 0.6 with \coordinate (x);
    %%}
%%}] plot[smooth] coordinates {(b) (3,-1) (1,-2) (c)};
%%\draw (x) node[below right] {$f\circ g^{-1}$};
%\draw[->,postaction={
    %decorate,
    %decoration={
        %markings,
        %mark=at position 0.6 with \coordinate (y);
    %}
%}] plot[smooth] coordinates {(a) (-.4,-.5) (-.1,-2) (c)};
%\draw (y) node[left] {Gauge bosons $A$};
%\draw[thick,->,postaction={
    %decorate,
    %decoration={
        %markings,
        %mark=at position 0.6 with \coordinate (z);
    %}
%}] plot[smooth] coordinates {(a) (2,.25) (3.5,0) (b)};
%\draw (z) node[above] {$g$};
%\fill (a) circle (1.5pt) node[above right] {$P$} (b) circle (1.5pt) node[above] {$g(P)$};
%\draw (-.1,.1) circle (0.8) ++(185:0.8) node[right] {$U$};
%\draw[very thick,rounded corners=3mm] (-1.5,0)--(-1.3,.5)--(-.8,1.2)--(0,1)--(1.5,1.2)--(1.75,.5)--(0.875,-.7)--(0,-1)--(-1,-.8)--cycle;
%\draw (-1.5,0) node[right] {$M$};
%\draw (3.9,-.1) circle (0.8);
%\draw[very thick] (3,1)--(5.5,1)--(5,-1.3)--(2.5,-1.3)--cycle;
%\draw (5.5,1) node[below left] {$R^n$};
%\draw[ultra thin] (4,-.5)--++(.2,-1) node[below] {$g(U)$};
%\end{tikzpicture}
\begin{tikzpicture}
\filldraw [fill=gray!10!white, draw=black] plot [smooth cycle] coordinates {(5,0.25) (6,0.35) (6.5, 0.2) (7,0.5) (7,1.65) (6.5,2.75) (5.8,2.75) (5.3,1.45) (4.8,0.85) } node at (6.2,1.7) {$(N, g)$} node at (6.2, 3.2) {Riemannian manifold};
\path[->] (2.6,1.7) edge [bend left] node[above] {Higgs field} node[below] {$\Phi$} (5.0,1.7);
%\filldraw [fill=gray!10!white, draw=black] plot [smooth cycle] coordinates {(1.0,.1)(1.5,.2)(2.8,.5)(2.9,1.5)(2.8,2.8)(1.4,2.5)(0.5,0.5)} node at (1.8,1.7) {$(M, \eta)$} node at (1.8, 3.2) {Spacetime};
\filldraw [fill=gray!10!white, draw=black] (-0.2,.5) to[bend left] (1.3,2.85) to[bend left] (2.7,2) to[bend right] (1.8,.5) to[bend right] cycle node at (1.2,1.7) {$(M, \eta)$} node at (1.2, 3.2) {Spacetime};
\path[->] (0.7,0.5) edge [bend right] node[left] {$A \in \Omega^1$} node[right] {Gauge bosons} (0.7, -1.5);
\filldraw [fill=gray, draw=black] (5.2, .7) circle (0.2);
\path[->] (2.5,-1.5) edge [bend right] node[right] {action $\gamma$} (5.2, .7);
\draw (-0.2,-1.7)-- (1.25, -1.7) node[below] {Lie algebra $\mathfrak{g}$} --(2.7,-1.7);
%\draw node at () {$\mathfrak{g}$};
%\draw node at (0,0.15) {$A \in \Omega^1$};
%\end{tikzpicture}
%\hspace*{11em}\raisebox{8em}{
%%\begin{flushright}
%\begin{tikzpicture}
\tikzset{shift={(5,-3)}} %Change position of the next drawing
\begin{axis}[ scale = 0.6,
            hide axis,
            %axis lines=middle,
%            axis on top,
%            axis line style={blue,dashed,thick},
%            ymin=-2,ymax=2,
%            xmin=-2,xmax=2,
%            zmin=-2,zmax=2,
            samples=40,
            domain=0:360,
            y domain=0:1.25,clip=false
        ]
        \addplot3 [surf, shader=flat, draw=black, fill=gray!10!white, z buffer=sort]
           ({sin(x)*y}, {cos(x)*y}, {(y^2-1)^2});
        \draw[blue,thick,dashed] (axis cs:0,0,0) -- (axis cs:1,0,0)
                    node[below,font=\footnotesize]{};
        \draw[blue,thick,-stealth] (axis cs:1,0,0) -- (axis cs:1.3,0,0)
                    node[above,font=\footnotesize]{};
        \draw[blue,thick,dashed] (axis cs:0,0,0) -- (axis cs:0,-1,0)
                    node[left=2mm,font=\footnotesize]{}; %{Label} am Ende 
        \draw[blue,thick,-stealth] (axis cs:0,-1,0) -- (axis cs:0,-1.5,0)
                    node[right=1mm,font=\footnotesize]{};
        \draw[blue,thick,dashed] (axis cs:0,0,0) -- (axis cs:0,0,1)
                    %node[left=2mm,font=\footnotesize]{$\phi_{\text{RE}}$}
                    ;
        \draw[blue,thick,-stealth] (axis cs:0,0,1) -- (axis cs:0,0,1.3)
                    node[right,font=\footnotesize]{$\mathcal{V}$};
        \end{axis}
\end{tikzpicture}
%\end{flushright}
\end{frame}

\begin{frame}{Guide: Curved Yang-Mills-Higgs gauge theory}
\setbeamercovered{invisible}
\begin{table}[h!]
		\begin{tabularx}{\textwidth}{X X}
			\rowcolor{gray}
			Classical formalism & CYMH GT \\
			Lie algebra $\mathfrak{g}$ as $M \times \mathfrak{g}$ & Lie algebroid $E \to N$ \\
			\rowcolor{Gray}
			$\mathfrak{g}$-action $\gamma$ & Anchor $\rho$ of $E$ \\ 
			\rowcolor{Gray}
			& \& $E$-connections \\
			Canonical flat connection $\nabla^0$ on $M \times \mathfrak{g}$ & General connection $\nabla$ on $E$
		\end{tabularx}
\end{table}
\pause
\begin{remark}[Why a "curved theory"?]
Usually, the field strength $F$ is given by (abelian, for simplicity)
\bas
F
&\coloneqq
\mathrm{d}A
=
\mathrm{d}^{\nabla^0}A.
\eas
$\rightsquigarrow$ We will use a general connection $\nabla$ instead of $\nabla^0$, and $\nabla$ may not be flat.
\end{remark}
\end{frame}
}

\subsection{Mathematical basics}

\renewcommand\insertreferences{{\tiny Ana Cannas Da Silva, Alan Weinstein. \textit{Geometric models for noncommutative algebras}, \newline volume 10. American Mathematical Society, 1999.}}

\begin{frame}
%Instead of Lie algebras we want to use:
\begin{definition}[Lie algebroids]
Let $E \to N$ be a vector bundle. Then $E$ is a Lie algebroid, if there is a bundle map $\rho: E \to \mathrm{T}N$, called the \textbf{anchor}, and a Lie algebra structure on $\Gamma(E)$ with Lie bracket $\mleft[\cdot, \cdot\mright]_E$ satisfying
\ba
  \mleft[\mu, f \nu\mright]_E = f \mleft[\mu, \nu\mright]_E + \mathcal{L}_{\rho(\mu)}(f) ~ \nu
\label{eq:E-Leibniz}
\ea
for all $f \in C^\infty(N)$ and $\mu, \nu \in \Gamma(E)$.
\end{definition}
\pause
\begin{example}
\begin{itemize}[<+->]
	\item $E = \mathrm{T}N$, $\rho = \mathds{1}_{\mathrm{T}N}$
	\item $E$ a bundle of Lie algebras, $\rho = 0$
\end{itemize}
\end{example}
\end{frame}
%
%\renewcommand\insertreferences{{\tiny Yvette Kosmann-Schwarzbach and Franco Magri. Poisson-Nijenhuis structures. \newline \textit{In Annales de l’IHP Physique théorique}, volume 53, pages 35–81, 1990.}}
%
%\begin{frame}
%As for an action $\gamma: \mathfrak{g} \to \mathfrak{X}(N)$ we have:
%\begin{proposition}[Anchor as homomorphism]
%$\rho: \Gamma(E) \to \mathfrak{X}(N)$ is a homomorphism of Lie algebras.
%\end{proposition}
%\end{frame}

\renewcommand\insertreferences{{\tiny Ana Cannas Da Silva, Alan Weinstein. \textit{Geometric models for noncommutative algebras}, \newline volume 10. American Mathematical Society, 1999.}}

\begin{frame}
The classical formalism will correspond to:
\begin{proposition}[Action Lie algebroids]
Let $\mleft(\mathfrak{g}, \mleft[\cdot, \cdot \mright]_{\mathfrak{g}}\mright)$ be a Lie algebra with a $\mathfrak{g}$-action $\gamma$ on $N$. Then there is a \textbf{unique} Lie algebroid structure on $E \coloneqq N \times \mathfrak{g}$ as a vector bundle over $N$ such that
\ba
\rho(\nu)
&=
\gamma(\nu),
\\
\mleft[\mu, \nu\mright]_E
&=
\mleft[\mu, \nu\mright]_{\mathfrak{g}}
\ea
for all constant sections $\mu, \nu \in \Gamma(E)$.
\end{proposition}
\end{frame}

{
\setbeamertemplate{footline}{}
\begin{frame}
\begin{table}[h!]
		\begin{tabularx}{\textwidth}{X X}
			\rowcolor{gray}
			Classical formalism & CYMH GT \\
			Lie algebra $\mathfrak{g}$ as $M \times \mathfrak{g}$ & Lie algebroid $E \to N$ \\
			\rowcolor{Gray}
			$\mathfrak{g}$-action $\gamma$ & Anchor $\rho$ of $E$ \\ 
			\rowcolor{Gray}
			& \& \textbf{$E$-connections} as lift of $\rho$ \\
			Canonical flat connection $\nabla^0$ on $M \times \mathfrak{g}$ & General connection $\nabla$ on $E$
		\end{tabularx}
\end{table}
			%\newline
			\pause
		$\rightsquigarrow$ $E$-connections similar to vector bundle connections but Leibniz rule along $\rho$, \textit{i.e.}~only lifting vector fields in the image of $\rho$
\end{frame}
}

\renewcommand\insertreferences{{ Mohamed Boucetta. Riemannian geometry of Lie algebroids. \newline \textit{Journal of the Egyptian Mathematical Society}, 19(1-2):57–70, 2011. }}

\begin{frame}
\begin{example}[$E$-connection ${}^E\nabla$ on $V$]
$\nabla^\prime$ a vector bundle connection on $V \to N$, then
\ba
{}^E\nabla_\nu v
&\coloneqq
\nabla^\prime_{\rho(\nu)} v
\ea
for all $\nu \in \Gamma(E)$ and $v \in \Gamma(V)$. In short denoted by $\nabla^\prime_\rho$.
\end{example}
\end{frame}

\renewcommand\insertreferences{{\tiny Camilo Arias Abad, Marius Crainic. Representations up to homotopy of Lie algebroids. \newline \textit{Journal für die reine und angewandte Mathematik (Crelles Journal)}, 2012(663):91–126, 2012.}}

\begin{frame}
\begin{example}
For $\nabla$ a connection on $E$ we have the \textbf{basic connection} given as a pair of $E$-connections on $E$ and on $\mathrm{T}N$ by
\ba
\nabla^{\mathrm{bas}}_\nu \mu
&\coloneqq
\mleft[ \nu, \mu \mright]_E
	+ \nabla_{\rho(\mu)} \nu,
\\
\nabla^{\mathrm{bas}}_\nu X
&\coloneqq
\mleft[ \rho(\nu), X \mright]
	+ \rho\mleft(\nabla_{X} \nu\mright)
\ea
for all $X \in \mathfrak{X}(N)$ and $\nu, \mu \in \Gamma(E)$.
\end{example}
\pause
\begin{remark}[Encoding of Lie algebra representations]
Test this with trivial bundles and canonical flat connection $\nabla^0$, \textit{i.e.}~$E = N \times \mathfrak{g}$ and $\nabla^0 \nu = 0$ for constant sections $\nu$.
\end{remark}
\end{frame}

%\renewcommand\insertreferences{{\tiny Camilo Arias Abad, Marius Crainic. Representations up to homotopy of Lie algebroids. \newline \textit{Journal für die reine und angewandte Mathematik (Crelles Journal)}, 2012(663):91–126, 2012.}}
\begin{frame}
\begin{definition}[Basic curvature]
Let $\nabla$ be a connection on $E$.
The \textbf{basic curvature $R_\nabla^{\mathrm{bas}}$} is defined as an element of $\Gamma\left(\bigwedge^2E^* \otimes \mathrm{T}^*N \otimes E \right)$ by
\ba
R^{\mathrm{bas}}_\nabla(\mu, \nu) X
&\coloneqq
\nabla_X\mleft(\mleft[\mu, \nu\mright]_E\mright) 
	- \mleft[ \nabla_X \mu, \nu \mright]_E 
	- \mleft[ \mu, \nabla_X \nu \mright]_E 
\nonumber\\
&\hspace{1cm}
	- \nabla_{\nabla^{\mathrm{bas}}_\nu X} \mu 
	+ \nabla_{\nabla^{\mathrm{bas}}_\mu X} \nu,
\ea
where $\mu, \nu \in \Gamma(E)$ and $X \in \mathfrak{X}(N)$.
\end{definition}

\pause
\begin{proposition}
We recover the curvature of the basic connection:
\ba
R_{\nabla^{\mathrm{bas}}}
&=
\mleft\{
\begin{array}{ll}
- R_\nabla^{\mathrm{bas}}(\cdot, \cdot) \circ \rho, & \text{on $E$}, \\
- \rho \circ R_\nabla^{\mathrm{bas}}, & \text{on $\mathrm{T}N$}.
\end{array}
\mright.
\ea
\end{proposition}
\end{frame}

\subsection{Curved Yang-Mills-Higgs gauge theory}

{
\makeatletter
\setbeamertemplate{footline}
{
  \leavevmode%
  \hbox{%
  \begin{beamercolorbox}[wd=.09\paperwidth,ht=2.25ex,dp=1ex,center]{author in head/foot}%
    \usebeamerfont{author in head/foot}
		Sources
  \end{beamercolorbox}%
  \begin{beamercolorbox}[wd=.91\paperwidth,ht=2.25ex,dp=1ex,center]{title in head/foot}%
    \usebeamerfont{title in head/foot}
		Alexei Kotov and Thomas Strobl. Curving Yang-Mills-Higgs gauge theories. \textit{Physical Review D}, 92(8):085032, 2015.
  \end{beamercolorbox}%
  %\begin{beamercolorbox}[wd=.175\paperwidth,ht=2.25ex,dp=1ex,right]{date in head/foot}%
    %\usebeamerfont{date in head/foot}\insertshortdate{}\hspace*{2em}
    %%\insertframenumber{} 
		%%/ \inserttotalframenumber\hspace*{2ex} 
  %\end{beamercolorbox}
	}%
  \vskip0pt%
}
\begin{frame}
\begin{definition}[Space of fields]
Fields are a pair consisting of:
\begin{itemize}[<+->]
	\item Higgs field $\Phi \in C^\infty(M;N)$
	\item Field of gauge bosons $A \in \Omega^1(M; \Phi^*E)$
\end{itemize}
\end{definition}
%\end{frame}
\pause
%\begin{frame}
\begin{definition}[Minimal coupling]
\textbf{Minimal coupling} $\mathfrak{D}$, $(\Phi, A) \mapsto \mathfrak{D}(\Phi, A) \in \Omega^1(M; \Phi^*\mathrm{T}N)$, by
\ba
\mathfrak{D}(\Phi, A)
&\coloneqq
\mathfrak{D}^A \Phi
\coloneqq
\mathrm{D}\Phi
	- (\Phi^*\rho)(A),
\ea
where $\mathrm{D}\Phi: \mathrm{T}M \to \mathrm{T}N$ is the tangent map.
\end{definition}
\end{frame}

\begin{frame}
\setbeamercovered{invisible}
\begin{definition}[Field strength]
Let $\nabla$ be a connection on $E$. We define the \textbf{field strength $F$}, $(\Phi, A) \mapsto F(\Phi, A) \in \Omega^2(M; \Phi^*E)$, by
\ba
F(\Phi, A)
\coloneqq
\mathrm{d}^{\Phi^*\nabla} A
	+ \frac{1}{2} \mleft( \Phi^* t_{\nabla^{\mathrm{bas}}} \mright)\mleft( A \stackrel{\wedge}{,} A \mright),
\ea
where $t_{\nabla^{\mathrm{bas}}}$ is the torsion of $\nabla^{\mathrm{bas}}$ on $E$ and $\mathrm{d}^{\Phi^*\nabla}$ the exterior covariant derivative of $\Phi^*\nabla$.
\end{definition}
\pause
\begin{definition}[Generalised field strength]
Let $\zeta$ be an element of $\Omega^2(N; E)$, then we define the \textbf{generalised field strength $\mathcal{F}$} by
\ba
\mathcal{F}(\Phi,A)
&\coloneqq
F(\Phi, A)
	+ \frac{1}{2} (\Phi^*\zeta)\mleft( \mathfrak{D}^A \Phi \stackrel{\wedge}{,} \mathfrak{D}^A \Phi \mright).
\ea
\end{definition}
\end{frame}

\begin{frame}
\begin{definition}[Curved Yang-Mills-Higgs (CYMH) Lagrangian]
Let $\kappa$ be a fibre metric on $E$, then the \textbf{curved Yang-Mills-Higgs Lagrangian $\mathfrak{L}_{\mathrm{CYMH}}$}, $(\Phi, A) \mapsto \mathfrak{L}_{\mathrm{CYMH}}(\Phi, A) \in \Omega^{\mathrm{dim}(M)}(M)$, is defined by
\ba
\mathfrak{L}_{\mathrm{CYMH}}(\Phi, A)
&\coloneqq
- \frac{1}{2} \mleft( \Phi^*\kappa \mright)\bigl(\mathcal{F}(\Phi, A) \stackrel{\wedge}{,} *\mathcal{F}(\Phi, A) \bigr)
\nonumber \\
&\hspace{1cm}
	+ \mleft( \Phi^*g \mright)\mleft(\mathfrak{D}^A \Phi \stackrel{\wedge}{,} *\mathfrak{D}^A \Phi \mright)
	- *(\Phi^*\mathcal{V}),
\ea
where $*$ is the Hodge star operator related to the spacetime metric $\eta$.
\end{definition}
\end{frame}

%Add slide about the aim of CYMH GT

\begin{frame}
\begin{definition}[CYMH GT]
Assume we have additionally the \textbf{compatibility conditions}
\ba
	R_\nabla + \mathrm{d}^{\nabla^{\mathrm{bas}}} \zeta &= 0,\label{EqMyFormulationOfZetaCondition} \\
	R_\nabla^{\mathrm{bas}} &= 0, \label{VanishingBasicCurvComp} \\
	\nabla^{\mathrm{bas}} \kappa &= 0, \\
	\nabla^{\mathrm{bas}} g &= 0, \\
	\mathcal{L}_\rho \mathcal{V} &= 0,
\ea
then we say that we have a \textbf{curved Yang-Mills-Higgs gauge theory}.
\pause

We say that we have a \textbf{pre-classical gauge theory}, if $\nabla$ is flat.
\pause

If we have additionally $\zeta \equiv 0$, then we say that we have a \textbf{classical gauge theory}.
\end{definition}
\end{frame}

\begin{frame}{Summary}
\begin{center}
	\begin{tikzcd}[ampersand replacement=\&, column sep=small]
		\Phi^*E \arrow{d} \& (E, \mleft[ \cdot, \cdot \mright]_E, \kappa, \nabla) \arrow{d} \arrow{rd}{\rho} \\
		M \arrow[r, "\Phi"] \& (N, g) \& \arrow{l} \mathrm{T}N \arrow[lu, bend right, swap, "\zeta \in \Omega^2"] 
	\end{tikzcd}
\end{center}

$\rightsquigarrow$ Together with the compatibility conditions we have gauge invariance, that is,
\ba
\delta_\varepsilon \mathfrak{L}_{\mathrm{CYMH}}
&=
0.
\ea
\pause
\begin{remark}[{[S.-R.\ F.]}]
If $\rho \equiv 0$, then $\varepsilon \in \Gamma(\Phi^*E)$ and
\bas
\delta_\varepsilon \Phi&=0,&
\delta_\varepsilon A&= \mleft[ \varepsilon, A \mright]_E -\mathrm{d}^{\Phi^*\nabla}\varepsilon.
\eas
\end{remark}
\end{frame}
}

\section{Field Redefinition}
\subsection{General definition}
{
\setbeamertemplate{footline}{}

\begin{frame}
\begin{motivation}
\begin{itemize}[<+->]
	\item Are there CYMH GTs which are neither pre-classical nor classical?
	\item Difficulty: There is an equivalence relation of CYMH GTs keeping the same Lagrangian and preserving the physics, possibly turning theories into (pre-)classical ones. 
	\newline
	
	This was provided by Edward Witten in a private communication with Thomas Strobl about a specific example of a CYMH GT.
\end{itemize}
\end{motivation}
\end{frame}

\begin{frame}
\begin{definition}[Field redefinition, {[S.-R. F.]}]
Let $\lambda \in \Omega^1(N; E)$ such that $\Lambda \coloneqq \mathds{1}_E - \lambda \circ \rho$ is an automorphism of $E$. We then define the \textbf{field redefinitions} by 
\ba\label{EqFieldRedefFuerA}
\widetilde{A}^\lambda
&\coloneqq 
\mleft( \Phi^* \Lambda \mright) (A)+ \Phi^! \lambda,
\\
\widetilde{\nabla}^\lambda
&\coloneqq
\nabla
	+ \mleft( \Lambda \circ \mathrm{d}^{\nabla^{\mathrm{bas}}} \circ \Lambda^{-1} \mright) \lambda,
\label{FieldTrafoOfNabla} \\
\widetilde{\kappa}^\lambda
&\coloneqq
\kappa \circ \mleft( \Lambda^{-1}, \Lambda^{-1} \mright),
\label{FieldTrafoOfKappa} \\
\widetilde{g}^\lambda
&\coloneqq
g \circ \mleft( \widehat{\Lambda}^{-1}, \widehat{\Lambda}^{-1} \mright), \label{FieldTrafoOfG}
\ea
where $\widehat{\Lambda} \coloneqq \mathds{1}_{\mathrm{T}N} - \rho \circ \lambda$, and for all $X, Y \in \mathfrak{X}(N)$ we also define
\ba
&\widetilde{\zeta}^\lambda\mleft(\widehat{\Lambda}(X),\widehat{\Lambda}(Y)\mright)
\nonumber \\
&\coloneqq
\Lambda\bigl(
	\zeta\mleft( X, Y \mright)
\bigr)
	- \mleft(\mathrm{d}^{\widetilde{\nabla}^\lambda} \lambda\mright)(X,Y)
	+ t_{\widetilde{\nabla}^\lambda_\rho}(\lambda(X), \lambda(Y)).
\ea
\end{definition}
\end{frame}

\begin{frame}
\begin{proposition}[{[S.-R. F.]}]
\begin{itemize}
	\item Field redefinitions define an equivalence relation of CYMH gauge theories
	\item $\widetilde{\mathfrak{L}}^\lambda_{\mathrm{CYMH}} = \mathfrak{L}_{\mathrm{CYMH}}$
\end{itemize}
\end{proposition}
%\vfill
\pause
Let us now apply a field redefinition in order to study whether $\nabla$ and $\zeta$ can become flat and zero, respectively.
\end{frame}
}

\subsection{Lie algebra bundles}

{
\setbeamertemplate{footline}{}
\begin{frame}{What happens in the case of Lie algebra bundles?}
\begin{example}[Lie algebra bundles (LABs)]
\begin{itemize}[<+->]
	\item $E = \mathcal{g}$ an LAB ($\rho \equiv 0$) with a field of Lie brackets $\mleft[ \cdot, \cdot \mright]_{\mathcal{g}} \in \Gamma\mleft( \bigwedge^2{\mathcal{g}}^* \otimes {\mathcal{g}} \mright)$ which restricts to the bracket of a given Lie algebra $\mathfrak{g}$
\end{itemize}
\pause
Compatibilities:
\begin{itemize}[<+->]
	\item $\kappa$ needs to be $\mathrm{ad}$-invariant
	\item We need
\ba
\nabla_Y\mleft( \mleft[ \mu, \nu \mright]_{\mathcal{g}} \mright)
&=
\mleft[ \nabla_Y \mu, \nu \mright]_{\mathcal{g}}
	+ \mleft[ \mu, \nabla_Y \nu \mright]_{\mathcal{g}}, \\
R_\nabla(Y, Z) \mu
&=
\mleft[ \zeta(Y, Z), \mu \mright]_{\mathcal{g}}
\ea
for all $Y, Z \in \mathfrak{X}(N)$ and $\mu, \nu \in \Gamma({\mathcal{g}})$.
\end{itemize}
\end{example}
\end{frame}
\begin{frame}
\begin{theorem}[Invariant for LABs, {[S.-R. F.]}]
We have
\ba	
\mathrm{d}^{\widetilde{\nabla}^\lambda} \widetilde{\zeta}^\lambda = \mathrm{d}^\nabla \zeta,
\ea
and $\mathrm{d}^\nabla \zeta$ has values in the centre of ${\mathcal{g}}$.
\end{theorem}
\end{frame}

\begin{frame}{Behaviour of the field redefinition of $\zeta$}
\begin{theorem}[Existence of non-classical theories, {[S.-R. F.]}]
If $\mathrm{d}^\nabla \zeta \neq 0$, then there is no field redefinition such that $\widetilde{\zeta}^\lambda = 0$.
\end{theorem}
\pause
\begin{remark}
Starting with a classical theory:

If $\mathrm{dim}(N) \geq 3$ and if Lie algebra $\mathfrak{g}$ has a non-zero centre, then we can always construct a pre-classical CYMH GT which is not a classical one by adding a $\zeta$ with $\mathrm{d}^\nabla \zeta \neq 0$.
\end{remark}
\pause
However, by $R_\nabla = \mathrm{ad}_{\mathcal{g}} \circ \zeta$ it may still be that $\nabla$ becomes flat.
\end{frame}

\begin{frame}{Turning to the field redefinition of $\nabla$:}
\begin{theorem}[Differential on centre-valued forms, {[S.-R. F.]}]
$\nabla$ restricts to the centre of ${\mathcal{g}}$ and induces a differential $\mathrm{d}^\Xi$ on centre-valued forms. Moreover, $\mathrm{d}^\Xi$ is independent of the field redefinitions.
\end{theorem}
\pause
\begin{proof}[Sketch of proof]
Recall
\bas
\nabla_Y\mleft( \mleft[ \mu, \nu \mright]_{\mathcal{g}} \mright)
&=
\mleft[ \nabla_Y \mu, \nu \mright]_{\mathcal{g}}
	+ \mleft[ \mu, \nabla_Y \nu \mright]_{\mathcal{g}}, \\
R_\nabla(Y, Z) \mu
&=
\mleft[ \zeta(Y, Z), \mu \mright]_{\mathcal{g}}, \\
\widetilde{\nabla}^\lambda_Y \mu
&=
\nabla_Y \mu
- \mleft[\lambda(Y), \mu\mright]_{\mathcal{g}},
\eas
for all $Y, Z \in \mathfrak{X}(N)$ and $\mu, \nu \in \Gamma({\mathcal{g}})$. Then insert $\mu$ with values in the centre.
\end{proof}
\end{frame}

\begin{frame}
\begin{theorem}[Closedness of $\mathrm{d}^\nabla \zeta$, {[S.-R. F.]}]
We have
\ba
\mathrm{d}^\Xi \mathrm{d}^\nabla \zeta
&=
0.
\ea
\end{theorem}
\pause
\begin{definition}[Obstruction class, {[S.-R. F.]}]
We define the \textbf{obstruction class} by
\ba
\mathrm{Obs}(\Xi)
&\coloneqq
\mleft[ \mathrm{d}^\nabla \zeta \mright]_{\mathrm{d}^\Xi}.
\ea
\end{definition}
\pause
\begin{proposition}[{[S.-R. F.]}]
\begin{itemize}[<+->]
	\item An invariant of the field redefinitions.
	\item If $\nabla$ flat, then $\mathrm{Obs}(\Xi) = 0$.
\end{itemize}
\end{proposition}
\end{frame}
}

{
\setbeamertemplate{footline}{}
\begin{frame}
\begin{theorem}[Obstruction for non-pre-classical theories, {[S.-R. F.]}]
If $\mathrm{Obs}(\Xi) \neq 0$, then there is no field redefinition such that $\widetilde{\nabla}^\lambda$ is flat.
\end{theorem}
\pause
\begin{theorem}[Locally always pre-classical]
If $N$ is contractible, then there is a field redefinition such that $\widetilde{\nabla}^\lambda$ is flat.
\end{theorem}
\pause
\begin{remark}
Second theorem follows as a result by K.~Mackenzie (General Theory of Lie Groupoids and Algebroids. \textit{London Mathematical Society Lecture Note Series}, 213, 2005). Mackenzie derived $\mathrm{Obs}(\Xi)$ in the context of extending Lie algebroids by LABs.
\end{remark}
\end{frame}
}

\renewcommand\insertreferences{{\tiny  K. Mackenzie. General Theory of Lie Groupoids and Algebroids. \newline \textit{London Mathematical Society Lecture Note Series}, 213, 2005.}}

\begin{frame}
\begin{example}[Zero obstruction class not necessarily pre-classical]
Let $P$ be the Hopf fibration
\begin{center}
	\begin{tikzcd}[ampersand replacement=\&, column sep=small]
		\mathrm{SU}(2) \arrow{r}	\& \mathds{S}^7 \arrow{d} \\
			\& \mathds{S}^4
	\end{tikzcd}
\end{center}
Then for the adjoint bundle
\bas
{\mathcal{g}}
&\coloneqq
P \times_{\mathrm{SU}(2)} \mathfrak{su}(2)
\coloneqq 
\mleft( \mathds{S}^7 \times \mathfrak{su}(2) \mright) \Big/ \mathrm{SU}(2)
\eas
we have a non-flat $\nabla$ satisfying the compatibility conditions such that all of its field redefinitions are not flat either, but $\mathrm{Obs}(\Xi) = 0$.
\end{example}
\end{frame}

{
\setbeamertemplate{footline}{}
\begin{frame}{Summary}
\begin{remark}
Locally, LABs are always pre-classical but not necessarily classical. 

In general, $\mathrm{Obs}(\Xi)=0$ does not imply a flat connection.
\end{remark}
\pause

So, what actually happens in the adjoint bundle of $\mathds{S}^7 \to \mathds{S}^4$? 

$\rightsquigarrow$ Integration
\end{frame}
}
\section{Integration (WIP)}
\subsection{Principal bundles based on Lie group bundle actions}
{
\setbeamertemplate{footline}{}

\begin{frame}{Idea}
	\begin{remark}
		Observe that the Higgs field $\Phi$ doesn't really matter at all in the case of LABs
		
		$\rightsquigarrow$ Think in terms of bundles over $M$ directly
	\end{remark}
	
\pause
\begin{table}[h!]
		\begin{tabularx}{\textwidth}{X| c c} 
			\rowcolor{gray}
			& Classical & Curved \\ \hline
			Infinitesimal & Lie algebra $\mathfrak{g}$ & LAB $\mathcal{g}$ \\
			\rowcolor{Gray}
			Integrated & Lie group $G$ & \textcolor[rgb]{1,0.41,0.13}{LGB\footnote{LGB = Lie group bundle} $\mathcal{G}$?} \\
		\end{tabularx}
\end{table}

\begin{center}
	\begin{tikzcd}[ampersand replacement=\&]
	G \arrow{r} \& \mathcal{G} \arrow{d}{\pi} \\
	\& M
	\end{tikzcd}
\end{center}

\end{frame}
}

\begin{frame}
\begin{definition}[LGB actions, simplified]
\begin{center}
	\begin{tikzcd}[ampersand replacement=\&, column sep = small, row sep = small]
	\& \mathcal{G} \arrow{d} \\
	\mathcal{P} \arrow{r}{\pi} \& M
	\end{tikzcd}
\end{center}
$\mathcal{P} \stackrel{\pi}{\to} M$ a fibre bundle. A \textbf{right-action of $\mathcal{G}$ on $\mathcal{P}$} is a smooth map 
%\bas
$\mathcal{P} * \mathcal{G} \coloneqq \pi^*\mathcal{G} = \mathcal{P} \times_M \mathcal{G} \to \mathcal{P}$,
$(p, g) \mapsto p \cdot g$,
%\eas
satisfying the following properties:
\ba\label{InvarianceOffUnderGAction}
\pi(p \cdot g) &= \pi(p),\\
(p \cdot g) \cdot h &= p \cdot (gh),\\
p \cdot e_{\pi(p)} &= p
\ea
for all $p \in \mathcal{P}$ and $g, h \in \mathcal{G}_{\pi(p)}$, where $e_{\pi(p)}$ is the neutral element of $\mathcal{G}_{\pi(p)}$.
\end{definition}
\end{frame}

{
\setbeamertemplate{footline}{}
\begin{frame}{Examples}
	\begin{example}
	$\mathcal{G}$ acts canonically on itself:
	\bas
	\mathcal{G} * \mathcal{G} &\to \mathcal{G},\\
	(q,h) &\mapsto qh.
	\eas
	\end{example}
\pause
\begin{example}[Recovering Lie group action]
$\bullet$ Either by $M = \{*\}$.

$\bullet$ Or by $\mathcal{G} \cong M \times G$, then also $\mathcal{P} * \mathcal{G} \cong \mathcal{P} \times G$, and we can define
\bas
\mathcal{P} \times G &\to \mathcal{P}, \\
(p, g) &\mapsto p \cdot g \coloneqq p \cdot \bigl( \pi(p), g \bigr),
\eas
which is equivalent to $\mathcal{P} * \mathcal{G} \to \mathcal{P}$.
\pause

$\Rightarrow$ Think of the "classical" theory as coming from a trivial LGB
\end{example}
\end{frame}
}

\renewcommand\insertreferences{{\tiny  Ieke Moerdijk, Janez Mrcun. Introduction to Foliations and Lie Groupoids. \newline \textit{Cambridge Studies in Advanced Mathematics 91, Cambridge University Press, Cambridge}, 2003}}

\begin{frame}
\begin{definition}[Principal bundle]
Still a fibre bundle
\begin{center}
	\begin{tikzcd}[ampersand replacement=\&]
	G \arrow{r} \& \mathcal{P} \arrow{d}{\pi} \\
	\& M
	\end{tikzcd}
\end{center}
but with $\mathcal{G}$-action
\bas
\begin{matrix}
	\textcolor[rgb]{1,0,0}{\xcancel{\mathcal{P} \times G}} &\to \mathcal{P} \\
	\mathcal{P} * \mathcal{G} &
\end{matrix}
\eas
simply transitive on fibres of $\mathcal{P}$, and "suitable" atlas.
\end{definition}
\end{frame}
\subsection{Connections}
{
\setbeamertemplate{footline}{}
\begin{frame}{Connection on $\mathcal{P}$: Idea}
\begin{figure}
%\caption{Pushforwards via right-multiplication of tangent vectors}  
%\label{figure:fibre bundle}
\centering
\begin{tikzpicture}[scale=0.55]
\draw (0,-3.5) to[out=10,in=170] (4.25,-3.5) to[out=350,in=190] (8.5,-3.5);
\draw (1.25,2.5) to[out=280,in=90](1.5,0) to[out=270,in=80] (1.25,-2.5);
\filldraw [fill=gray!20!white, draw=white] (3.5,2.5) to[out=280,in=90](3.75,0) to[out=270,in=80] (3.5,-2.5) -- (5,-2.5)  to[out=80,in=270](5.25,0) to[out=90,in=280] (5,2.5) -- cycle;
\draw (4.25,2.5) to[out=280,in=95](4.41,1) to[out=275,in=85] (4.41,-1) to[out=265,in=80] (4.25,-2.5); %draw with middle section for precise point placement

\draw (4,-1) node {\textcolor[rgb]{1,0,0}{$p$}};
\draw [<-, draw=blue] (5,1) to[out=275,in=85] (5,-1);
\draw (5.5,0) node {\textcolor[rgb]{0,0,1}{$\cdot g$}};
\filldraw [fill=red, draw=red] (4.41,1) circle (1pt);
\filldraw [fill=red, draw=red] (4.41,-1) circle (1pt);
\draw (5.75,2.5) to[out=280,in=90](6,0) to[out=270,in=80] (5.75,-2.5);
\draw (7.25,2.5) to[out=280,in=90](7.5,0) to[out=270,in=80] (7.25,-2.5);
\draw (2.75,2.475) to[out=280,in=90](3,0) to[out=270,in=80] (2.75,-2.475);

%\draw (4.75,2) node {$\mathcal{P}_x$};
%\draw (4.95,1.5) node {$\cong G$};
\draw (9.5,-3.5) node {$M$};
\draw (5,2) node {$\mathcal{P}_U$};
\draw (3.5,-3.4) node {$($};
\filldraw [fill=red, draw=red] (4.25,-3.5) circle (1pt);
\draw (4.25,-3.9) node {\textcolor[rgb]{1,0,0}{$x$}};
\draw (5,-3.6) node {$)$};
\draw (4.25, -3) node {$U$};
\draw (0,0) node {$\mathcal{P}$};
\draw [->] (0,-0.4) -- (0,-3.1);
\draw (0.4,-1.75) node {$\pi$};
\draw (3.5,1) node {\textcolor[rgb]{1,0,0}{$p \cdot g$}};
%%%%%%%%%%%%%%%%%%%%%%%%%%%%%%%%%%%%%%%%%%%%%%%%%%%%%%%%%%%%%%%%%%%%%%%%%%%%%%%%%%%%
\draw (10.5,-3.5) to[out=10,in=170] (14.75,-3.5) to[out=350,in=190] (19,-3.5); %hori M
\filldraw [fill=gray!20!white, draw=white] (14,2.5) to[out=280,in=90](14.25,0) to[out=270,in=80] (14,-2.5) -- (15.5,-2.5)  to[out=80,in=270](15.75,0) to[out=90,in=280] (15.5,2.5) -- cycle; %gray area
\draw (11.75,2.5) to[out=280,in=90](12,0) to[out=270,in=80] (11.75,-2.5); %vert line 1
\draw (14.75,2.5) to[out=280,in=90](15,0) to[out=270,in=80] (14.75,-2.5); %3
\draw (16.25,2.5) to[out=280,in=90](16.5,0) to[out=270,in=80] (16.25,-2.5); %4
\draw (17.75,2.5) to[out=280,in=90](18,0) to[out=270,in=80] (17.75,-2.5); %5

\draw (13.25,2.475) to[out=280,in=90](13.5,0) to[out=270,in=80] (13.25,-2.475); %2
\filldraw [fill=blue, draw=blue] (15,0) circle (1pt);
\draw (15.5,0) node {\textcolor[rgb]{0,0,1}{$g$}};

\draw (19,0) node {$\mathcal{G}$}; %These three lines are for LGB projection
\draw [->] (19,-0.4) -- (19,-3.1);
%\draw (18.5,-1.75) node {$\pi_{\mathcal{G}}$};

\draw (15.5,2) node {$\mathcal{G}_U$};
\draw (14,-3.4) node {$($};
\filldraw [fill=red, draw=red] (14.75,-3.5) circle (1pt);
\draw (14.75,-3.9) node {\textcolor[rgb]{1,0,0}{$x$}};
\draw (15.5,-3.6) node {$)$};
\draw (14.75, -3) node {$U$};

\path[<-] (8.5,0) edge [bend left] (11,0);
\end{tikzpicture}
\end{figure}
\pause
But:
\bas
&&&r_g: \mathcal{P}_x \to \mathcal{P}_x\\
&\Rightarrow&&\mathrm{D}_pr_g \text{ only defined on vertical structure}
\eas
\end{frame}

\begin{frame}{Connection on $\mathcal{P}$: Idea}
\begin{figure}
%\caption{Pushforwards via right-multiplication of tangent vectors}  
%\label{figure:fibre bundle part two}
\centering
\begin{tikzpicture}[scale=0.55]
\draw (0,-3.5) to[out=10,in=170] (4.25,-3.5) to[out=350,in=190] (8.5,-3.5);
\draw (1.25,2.5) to[out=280,in=90](1.5,0) to[out=270,in=80] (1.25,-2.5);
\filldraw [fill=gray!20!white, draw=white] (3.5,2.5) to[out=280,in=90](3.75,0) to[out=270,in=80] (3.5,-2.5) -- (5,-2.5)  to[out=80,in=270](5.25,0) to[out=90,in=280] (5,2.5) -- cycle;
\draw (4.25,2.5) to[out=280,in=95](4.41,1) to[out=275,in=85] (4.41,-1) to[out=265,in=80] (4.25,-2.5); %draw with middle section for precise point placement

\draw (4,-1) node {\textcolor[rgb]{1,0,0}{$p$}};
\draw [<-, draw=blue] (5,1) to[out=275,in=85] (5,-1);
\draw [<-, draw=blue] (4.8,1) to[out=275,in=85] (4.8,-1);
\draw [<-, draw=blue] (4.6,1) to[out=275,in=85] (4.6,-1);
\draw (5.5,0) node {\textcolor[rgb]{0,0,1}{$\cdot \sigma$}};
\filldraw [fill=red, draw=red] (4.41,1) circle (1pt);
\filldraw [fill=red, draw=red] (4.41,-1) circle (1pt);
\draw (5.75,2.5) to[out=280,in=90](6,0) to[out=270,in=80] (5.75,-2.5);
\draw (7.25,2.5) to[out=280,in=90](7.5,0) to[out=270,in=80] (7.25,-2.5);
\draw (2.75,2.475) to[out=280,in=90](3,0) to[out=270,in=80] (2.75,-2.475);

%\draw (4.75,2) node {$\mathcal{P}_x$};
%\draw (4.95,1.5) node {$\cong G$};
\draw (9.5,-3.5) node {$M$};
\draw (5,2) node {$\mathcal{P}_U$};
\draw (3.5,-3.4) node {$($};
\filldraw [fill=red, draw=red] (4.25,-3.5) circle (1pt);
\draw (4.25,-3.9) node {\textcolor[rgb]{1,0,0}{$x$}};
\draw (5,-3.6) node {$)$};
\draw (4.25, -3) node {$U$};
\draw (0,0) node {$\mathcal{P}$};
\draw [->] (0,-0.4) -- (0,-3.1);
\draw (0.4,-1.75) node {$\pi$};
\draw (3.5,1) node {\textcolor[rgb]{1,0,0}{$p \cdot \sigma_x$}};
%%%%%%%%%%%%%%%%%%%%%%%%%%%%%%%%%%%%%%%%%%%%%%%%%%%%%%%%%%%%%%%%%%%%%%%%%%%%%%%%
\draw (10.5,-3.5) to[out=10,in=170] (14.75,-3.5) to[out=350,in=190] (19,-3.5); %hori M
\filldraw [fill=gray!20!white, draw=white] (14,2.5) to[out=280,in=90](14.25,0) to[out=270,in=80] (14,-2.5) -- (15.5,-2.5)  to[out=80,in=270](15.75,0) to[out=90,in=280] (15.5,2.5) -- cycle; %gray area
\draw (11.75,2.5) to[out=280,in=90](12,0) to[out=270,in=80] (11.75,-2.5); %vert line 1
\draw (14.75,2.5) to[out=280,in=90](15,0) to[out=270,in=80] (14.75,-2.5); %3
\draw (16.25,2.5) to[out=280,in=90](16.5,0) to[out=270,in=80] (16.25,-2.5); %4
\draw (17.75,2.5) to[out=280,in=90](18,0) to[out=270,in=80] (17.75,-2.5); %5

\draw (13.25,2.475) to[out=280,in=90](13.5,0) to[out=270,in=80] (13.25,-2.475); %2
\draw [draw=blue] (14.25,0) to[out=10,in=170] (15,0) to[out=350,in=190] (15.75,0);
%\filldraw [fill=blue, draw=blue] (15,0) circle (1pt);
\draw (13.9,0) node {\textcolor[rgb]{0,0,1}{$\sigma$}};

\draw (19,0) node {$\mathcal{G}$}; %These three lines are for LGB projection
\draw [->] (19,-0.4) -- (19,-3.1);
%\draw (18.5,-1.75) node {$\pi_{\mathcal{G}}$};

\draw (15.5,2) node {$\mathcal{G}_U$};
\draw (14,-3.4) node {$($};
\filldraw [fill=red, draw=red] (14.75,-3.5) circle (1pt);
\draw (14.75,-3.9) node {\textcolor[rgb]{1,0,0}{$x$}};
\draw (15.5,-3.6) node {$)$};
\draw (14.75, -3) node {$U$};

\path[<-] (8.5,0) edge [bend left] (11,0);
\end{tikzpicture}
\end{figure}

\bas
\text{Use } \sigma \in \Gamma(\mathcal{G}): r_\sigma(p) \coloneqq p \cdot \sigma_{\pi(p)}
\eas
\end{frame}

\begin{frame}{Connection on $\mathcal{P}$: Revisiting the classical setup}
\setbeamercovered{invisible}
	\begin{minipage}[]{0.45\textwidth} 
	If $\mathcal{P}$ a typical principal bundle and \textcolor[rgb]{0,0.58,0}{$H$} a connection on it:
	\begin{figure}
	\begin{tikzpicture}[scale=1.2]
		%Gray area
		\filldraw [fill=gray!20!white, draw=white] (3.5,2.5) to[out=280,in=95](3.66,1) to[out=275,in=85] (3.66,-1) to[out=265,in=80] (3.5,-2.5) -- (5,-2.5) to[out=80, in=265] (5.16, -1) to[out=85, in=275] (5.16, 1) to[out=95,in=280] (5,2.5) -- cycle;
		%lines
		\draw (4.25,2.5) to[out=280,in=95](4.41,1) to[out=275,in=85] (4.41,-1) to[out=265,in=80] (4.25,-2.5);
		\definecolor{darkgreen}{rgb}{0,0.58,0}
		\draw [draw=darkgreen] (3.66,-1) to[out=280,in=200] (4.41,-1) to[out=20,in=100] (5.16, -1);
		\draw [draw=darkgreen] (3.66,1) to[out=280,in=200] (4.41,1) to[out=20,in=100] (5.16,1);
		%Auxiliary stuff like arrows and dots
		\filldraw [fill=red, draw=red] (4.41,1) circle (1pt);
		\filldraw [fill=red, draw=red] (4.41,-1) circle (1pt);
		\draw [<-, draw=blue] (5.16,1) to[out=275,in=85] (5.16,-1);
		\draw [->, thick] (4.41,-1) -- (4.71,-0.88);
		%\draw [->] (4.41,1) -- (4.63,2);
		\draw [->, thick] (4.41,1) -- (4.71,1.12);
		%\draw [<-,dotted,thick] (4.63,2) -- (4.71,1.12);
		%Labels
		\draw (3.2,1) node {\textcolor[rgb]{0,0.58,0}{$H_{p \cdot g}$}};
		\draw (3.2,-1) node {\textcolor[rgb]{0,0.58,0}{$H_p$}};
		\draw (5.5,0) node {\textcolor[rgb]{0,0,1}{$\cdot g$}};
		\draw (3,2.2) node {$\mathcal{P}_U$};
		\draw (4.71,-0.6) node {$X$};
		\draw (5,1.4) node {$\mathrm{D}_pr_g(X)$};
		%\draw (5.5,1.56) node {$\thicksim \mathrm{D}_x\sigma(X)$};
		%Text
	\end{tikzpicture}
	\end{figure}
	\end{minipage}\hfill
	\pause
	\begin{minipage}[]{0.45\textwidth} 
		$\mathcal{G}$ trivial, $\sigma \equiv g$ constant:
		\bas
		\mathrm{D}_p r_g (X)
		&=
		\mleft.\frac{\mathrm{d}}{\mathrm{d}t}\mright|_{t=0}\mleft( \alpha \cdot g \mright),
		\eas
		where $\alpha$ is the flow of $X$
	\end{minipage}
\end{frame}

\begin{frame}{Connection on $\mathcal{P}$: Revisiting the classical setup}
\setbeamercovered{invisible}
	\begin{minipage}[]{0.45\textwidth} 
	If $\mathcal{P}$ a typical principal bundle and \textcolor[rgb]{0,0.58,0}{$H$} a connection on it:
	\begin{figure}
	\begin{tikzpicture}[scale=1.2]
		%Gray area
		\filldraw [fill=gray!20!white, draw=white] (3.5,2.5) to[out=280,in=95](3.66,1) to[out=275,in=85] (3.66,-1) to[out=265,in=80] (3.5,-2.5) -- (5,-2.5) to[out=80, in=265] (5.16, -1) to[out=85, in=275] (5.16, 1) to[out=95,in=280] (5,2.5) -- cycle;
		%lines
		\draw (4.25,2.5) to[out=280,in=95](4.41,1) to[out=275,in=85] (4.41,-1) to[out=265,in=80] (4.25,-2.5);
		\definecolor{darkgreen}{rgb}{0,0.58,0}
		\draw [draw=darkgreen] (3.66,-1) to[out=280,in=200] (4.41,-1) to[out=20,in=100] (5.16, -1);
		\draw [draw=darkgreen] (3.66,1) to[out=280,in=200] (4.41,1) to[out=20,in=100] (5.16,1);
		%Auxiliary stuff like arrows and dots
		\filldraw [fill=red, draw=red] (4.41,1) circle (1pt);
		\filldraw [fill=red, draw=red] (4.41,-1) circle (1pt);
		\draw [<-, draw=blue] (5.16,1) to[out=275,in=85] (5.16,-1);
		\draw [->, thick] (4.41,-1) -- (4.71,-0.88);
		\draw [->, thick] (4.41,1) -- (4.63,2);
		\draw [->,dotted,thick] (4.41,1) -- (4.71,1.12);
		\draw [<-,dotted,thick] (4.63,2) -- (4.71,1.12);
		%Labels
		\draw (3.2,1) node {\textcolor[rgb]{0,0.58,0}{$H_{p \cdot \sigma_x}$}};
		\draw (3.2,-1) node {\textcolor[rgb]{0,0.58,0}{$H_p$}};
		\draw (5.5,0) node {\textcolor[rgb]{0,0,1}{$\cdot \sigma$}};
		\draw (3,2.2) node {$\mathcal{P}_U$};
		\draw (4.71,-0.6) node {$X$};
		\draw (5,2.25) node {$\mathrm{D}_pr_\sigma(X)$};
		\draw (5.5,1.56) node {$\thicksim \mathrm{D}_x\sigma_\pi(X)$};
		%Text
	\end{tikzpicture}
	\end{figure}
	\end{minipage}\hfill
	\begin{minipage}[]{0.45\textwidth} 
	%Properties regarding the shift:
	Now:
	\bas
	\mathrm{D}_p r_g(X) + \dots
		&=
		\mleft.\frac{\mathrm{d}}{\mathrm{d}t}\mright|_{t=0}\mleft( \alpha \cdot \sigma_{\pi \circ \alpha} \mright)
	\eas
	\pause
	\begin{itemize}
		\item Only vertical part in $\mathcal{G}$ of $\mathrm{D}\sigma$ matters
		\item Shift is vertical in $\mathcal{P}$
	\end{itemize}
	\pause
	\begin{remark}[General situation]
	Introduce connection on $\mathcal{G}$
	
	$\Rightarrow$ $\nabla$ on the LAB $\mathcal{g}$ of $\mathcal{G}$
	\end{remark}
	\end{minipage}
\end{frame}
%
%\begin{frame}{Horizontal distribution, definition}
	%\begin{definition}[Connection, {[S.-R.\ F.]}]
	%Let $H$ be a horizontal bundle of $\mathcal{P}$. Then $H$ is called a \textbf{connection on $\mathcal{P}$} if
	%\ba
	%\mleft( \mathrm{D}_p r_{\sigma} - \mleft.\mleft(\pi_{\mathcal{P}}^!\widetilde{\Delta \sigma}\mright)\mright|_{p\cdot \sigma_{\pi_{\mathcal{P}}(p)} } \mright)
	%\mleft( H_p \mright)
	%&=
	%H_{p\cdot \sigma_{\pi_{\mathcal{P}}(p)} }
	%\ea
	%for all $\sigma \in \Gamma(\mathcal{G})$,
	%where $\widetilde{\Delta \sigma}$ is the fundamental vector field of $\Delta\sigma$ which is the \textbf{Darboux derivative of $\sigma$} given by
	%\ba
	%\Delta\sigma
	%&\coloneqq
	%\sigma^! \mu_{\mathcal{G}}.
	%\ea
	%\end{definition}
%\end{frame}
%
%\begin{frame}
%\begin{example}
%Let $P$ be a $G$-principal bundle, equipped with a "classical" connection. 
%\pause
%
%$\Rightarrow$ Horizontal distribution on associated fibre bundles.
%\pause
%
%$\Rightarrow$ Horizontal distribution on the inner LGB $c(P)$ of $P$ given by
%\ba
%c(P) \coloneqq (P \times G) \Big/ G,
%\ea
%where $G$ acts on itself by conjugation.
%\pause
%
%$\Rightarrow$ Connection on $\mathcal{P} = c(P)$.
%\end{example}
%\end{frame}
%}

\section{Conclusion}
{
\setbeamertemplate{footline}{}
\begin{frame}{Summary}
\begin{table}[h!]
		\begin{tabularx}{\textwidth}{X| c c} 
			\rowcolor{gray}
			& Locally & Globally \\ \hline
			Curved Yang-Mills & Pre-classical & $\mathrm{ad}(\mathds{S}^7 \to \mathds{S}^4)$ curved 
			%\\
			%\rowcolor{Gray}
			%Tangent bundles & Pre-classical & $\mathrm{T}\mathds{S}^7$ curved \\
		\end{tabularx}
\end{table}
\begin{remark}[Integrated point of view]
This is probably linked to that an LGB is locally trivial 

$\rightsquigarrow$ LGB action locally equivalent to a Lie group action
\end{remark}

\end{frame}

\begin{frame}{Hope: Structural Lie groupoids}
%\begin{table}[h!]
		%\begin{tabularx}{\textwidth}{X| c c} 
			%\rowcolor{gray}
			%Gauge theory & Minimal coupling evolves around... \\ \hline
			%Yang-Mills & $\mathrm{GL}(V) \curvearrowright V$ \\
			%& $V$ vector space \\
			%\rowcolor{Gray}
			%Curved Yang-Mills & $\mathrm{Aut}_{\mathrm{base}}(V) \curvearrowright V$ \\ 
			%\rowcolor{Gray}
			%& $V$ vector bundle, $\mathrm{Aut}_{\mathrm{base}}$ base-preserving \\
			%Curved Yang-Mills-Higgs & $\mathrm{Aut}_{\neg\mathrm{base}}(V) \curvearrowright V$ \\
			%& $V$ vector bundle, $\mathrm{Aut}_{\mathrm{base}}$ base-preserving
		%\end{tabularx}
%\end{table}
\begin{table}[h!]
		\begin{tabularx}{\textwidth}{X| c c} 
			\rowcolor{gray}
			Gauge theory & Structure \\ \hline
			Yang-Mills & Lie group $G$ \\
			\rowcolor{Gray}
			Curved Yang-Mills & Lie group bundle $\mathcal{G}$ \\ 
			Curved Yang-Mills-Higgs & Lie groupoid $\mathfrak{G}$?
		\end{tabularx}
\end{table}

\begin{center}
	\begin{tikzcd}[ampersand replacement=\&, column sep=small]
		\& \arrow{ld} \mathfrak{G} \arrow{rd} \& \\
		M \arrow{rr}{\Phi} \& \& N
	\end{tikzcd}
\end{center}

\begin{remark}
\begin{itemize}
	\item Richer set of principal bundles, containing LGBs.
	\item May result into obstruction statements for curved Yang-Mills-Higgs gauge theories.
\end{itemize}
\end{remark}

\end{frame}
}
\thispagestyle{empty}
\topmargin -3.46 cm
\vspace*{\fill}
\begin{center}
\huge \textbf{Thank you!}
\end{center}
\vspace*{\fill}

\end{document}

